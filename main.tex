\documentclass[windows]{dlmudoctrans}
% 这个地方也可以参考 dlmuthesis 中的命令用法,实际上这部分代码就是从 dlmuthesis
% 里面迁移过来的。链接地址在:<https://github.com/JohnsonLo00/dlmuthesis>。
%
% 简单的来说,`\entitle' 对应的花括号里面写论文翻译以前原来的标题;`\cntitle'
% 里面写翻译之后的中文标题。然后估算一下自己的标题排版到封面上之后占据几行的下
% 划线,估计出来是三行,就在前面的方括号 `[  ]' 里面写 3;如果估计出来是两行,
% 就在前面的方括号 `[  ]' 里面写 2,以此类推。由于显示效果的限制,模板并不会根
% 据你给出的文本长度动态调整行数的指定,换句话说,最初编译出来的文档最初编译出
% 来的文档可能会有格式问题,也就是说,可能会因为指定的行数不够出现文字重叠的情
% 况。对于这种情况,你需要再回来手动加行数,然后再编译。
\entitle[3]{
  Unofficial \LaTeX Template for Translation of Foreign Language Documents for
  Undergraduate students of Dalian Maritime University
}
\cntitle[2]{非官方大连海事大学本科生毕业外文文献翻译\LaTeX{}模板}
\translator{你的名字}
\sdtID{2220XXXXXX}
\major{这里写专业年级班级}

\author[a]{张三}
\author[b]{李四}
\author[c, *]{Jack Green}
\author[d]{John Smith}

\affil[a]{国内某知名大学,\ XX市,\ XX省,000000}
\affil[b]{国内某研究机构,国内某大学,\ XX市,\ XX省,000000}
\affil[c]{A Famous Foreign University,\ City,\ Country}
\affil[d]{A Famous Foreign University,\ City,\ Country}
\affil[*]{通讯作者}

% 导入一个 BIB 文件,用来载入参考文献的信息。BIB 文件的写法可以自己早参考资料,
% 但大部分英文文献平台都支持直接将参考文献条目导出为 BIB 引用格式。如果你再线上
% 文献平台找到了你要翻译的文献,一般都可以直接在文献对应的“参考”页面下面找到
% 所有参考文献的列标及每个文件对应的链接。
\addbibresource{refs.bib}
% 这里还需要额外提醒用户注意一下:在 dlmuthesis 项目中,引用文献采用的是
% `natbib',但是本项目中却是 `biblatex'。这是出于很多原因的考量,此不赘述了。

% 下面这段按代码是用于展示批注功能的,如果不使用批注功能,可以暂时注释掉。我觉
% 得对于本科文献翻译来说,应该是用不到这个功能。
\let\comment\undefined
\usepackage[]{changes}
\definechangesauthor[name={阿海}, color=blue]{阿海} % 创建批注者信息
\definechangesauthor[name={海老师}, color=red]{海老师} % 创建批注者信息
% 但是要当心,因为在示例的章节 `mainbody/ch4.tex' 当中展示到了这个功能,所以直
% 接注释掉,如果不删掉第四章里的示例内容,会导致示例文档编译不通过。不过不影响
% 文档类本身的使用。
\begin{document}

% `\makepages' 用来生成符合要求的封面格式。
\makepages % 封面页
\maketocpages % 目录页

% 初始化正文部分格式并打印正文大标题
\cntitleinMAINBODY

% 这里是论文的摘要部分。你可以在这里写你的摘要,或者单独另外新建一个“*.tex”文
% 件,然后使用 `\input{}' 命令导入文件里的内容,都是可以的。视具体情况而定。
\begin{abstract}
  本文针对手机摄像头所获取的视频文件,进行手势运动方向的检测。针对低端摄像头视
  频图像的特点,本文采用了基于背景去除和肤色模型的方法对手部区域进行检测,并判
  别手部运动的方向。

  首先,获取视频图像序列,即从视频文件中获取每一帧图像作为待检测的视频图像序
  列;其次,对获取的视频图像序列中的每一帧图像进行颜色模型转换、背景去除、图像
  二值化、形态学处理等预处理;然后,利用区域增长方法来检测视频图像序列中的手部连
  通区域,并计算每帧图像中手部区域的中心;最后根据图像序列中手部区域中心位置的
  变化来判断手部运动方向。

  本文在Visual c++6.0开发环境下,借助于OpenCV开放平台,设计并实现了基于低端摄像
  头视频手势运动检测系统,得到了较好的检测效果。

  \keywords{\LaTeX{}, 文档类, 模板}
\end{abstract}

\tableofcontents\thispagestyle{plain}
\clearpage % 这个换页命令是格式要求,不要删除。

% `\pesudohookOFpremainbody' 命令是用来设置正文格式的,因为文档类设计的原因,
% `abstract' 环境开始时有一个用于初始化的钩子,会变更诸如页眉页脚之类的设置,但
% 是正文和摘要之间隔着一个目录,又不能在摘要环境的末尾直接加钩子,否则目录页的
% 格式就会变成和正文一样,这是不符合要求的。所以需要这个命令改回来。除此之外,
% 这个命令还会生成全文正文部分最开头出现一次的文章中文大标题,所以需要放在这个
% 位置 (目录页后、正文开始之前),不要删掉。
\pesudohookOFpremainbody
\input{mainbody/ch1} \clearpage
\input{mainbody/ch2} \clearpage
\input{mainbody/ch3} \clearpage
\input{mainbody/ch4} \clearpage

\begin{conclusion}

在Visual c++6.0开发环境下,借助于OpenCV开放平台,设计并实现了基于低端摄像头视频手势运动检测系统。

\end{conclusion}

 \clearpage % 结论

\printbibliography[heading=dlmustyled] % 参考文献
\end{document}
